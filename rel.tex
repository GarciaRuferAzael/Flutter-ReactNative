\documentclass[a4paper,12pt]{article}
\usepackage[utf8]{inputenc}
\usepackage{graphicx}
\usepackage{hyperref}
\usepackage{booktabs} % per tabelle più professionali
\usepackage{titlesec} % per formattare titoli
\usepackage{tocbibind} % per includere l'indice
\titleformat{\section}{\Large\bfseries}{}{0em}{}

\title{\textbf{Flutter vs React Native}}
\author{Azael Garcia Rufer}
\date{\today}

\begin{document}
\maketitle

\tableofcontents
\newpage

\section{Introduzione}
Negli ultimi anni, lo sviluppo di applicazioni mobili si è evoluto grazie a framework cross-platform come \textbf{Flutter} (Google) e \textbf{React Native} (Meta/Facebook). Questi strumenti permettono di creare app per Android e iOS con un’unica codebase, semplificando il processo di sviluppo.

Questa relazione si propone di:
\begin{itemize}
    \item Presentare Flutter e React Native, analizzandone vantaggi e svantaggi.
    \item Confrontarli in termini di performance, architettura e usabilità.
    \item Fornire indicazioni pratiche su quale scegliere in base alle esigenze del progetto.
\end{itemize}

\section{Flutter}
\subsection{Cos’è Flutter?}
Flutter è un framework open-source sviluppato da \textbf{Google} nel 2017. Utilizza il linguaggio \textbf{Dart} e un proprio motore di rendering chiamato \textbf{Skia}.

\subsection{Caratteristiche principali}
\begin{itemize}
    \item \textbf{Rendering Proprio}: Non usa i componenti nativi, ma disegna la UI con il proprio motore grafico.
    \item \textbf{Hot Reload}: Permette aggiornamenti istantanei senza riavviare l’app.
    \item \textbf{UI Personalizzabile}: Supporta Material Design (Android) e Cupertino (iOS).
    \item \textbf{Alta Performance}: Compilazione nativa (AOT - Ahead Of Time).
\end{itemize}

\subsection{Vantaggi e Svantaggi}
\textbf{Vantaggi:}
\begin{itemize}
    \item Prestazioni elevate e animazioni fluide.
    \item Interfaccia coerente tra iOS e Android.
    \item Supporto per Web, Desktop e Mobile.
\end{itemize}
\textbf{Svantaggi:}
\begin{itemize}
    \item App più pesanti rispetto a React Native.
    \item Dart meno diffuso rispetto a JavaScript.
\end{itemize}

\section{React Native}
\subsection{Cos’è React Native?}
React Native è un framework sviluppato da \textbf{Meta (Facebook)} nel 2015. Utilizza \textbf{JavaScript} e il framework React per creare app mobili sfruttando componenti nativi.

\subsection{Caratteristiche principali}
\begin{itemize}
    \item \textbf{Basato su JavaScript}: Facile da apprendere per chi conosce il web.
    \item \textbf{Componenti Nativi}: Usa gli elementi della UI propri di iOS e Android.
    \item \textbf{Hot Reload}: Permette di testare rapidamente le modifiche.
\end{itemize}

\subsection{Vantaggi e Svantaggi}
\textbf{Vantaggi:}
\begin{itemize}
    \item Ampia community e molte librerie disponibili.
    \item Facile integrazione con codice nativo.
    \item JavaScript è molto popolare tra gli sviluppatori.
\end{itemize}
\textbf{Svantaggi:}
\begin{itemize}
    \item Performance inferiore rispetto a Flutter (a causa del bridge JavaScript-to-Native).
    \item Problemi di compatibilità tra versioni di iOS e Android.
\end{itemize}

\section{Confronto Diretto}
\begin{table}[h]
    \centering
    \begin{tabular}{lcc}
        \toprule
        \textbf{Caratteristica} & \textbf{Flutter} & \textbf{React Native} \\
        \midrule
        Linguaggio & Dart & JavaScript (React) \\
        Rendering UI & Skia (personalizzato) & Componenti nativi \\
        Performance & Alta (compilazione AOT) & Buona, ma dipende dal bridge \\
        Facilità di apprendimento & Media & Facile (JavaScript è popolare) \\
        Community & In crescita & Più ampia \\
        Ecosistema & In espansione & Maturo e consolidato \\
        \bottomrule
    \end{tabular}
    \caption{Confronto tra Flutter e React Native}
\end{table}

\section{Quale Scegliere?}
\textbf{Usa Flutter se:}
\begin{itemize}
    \item Vuoi prestazioni elevate e una UI consistente.
    \item Hai bisogno di sviluppare anche per Web e Desktop.
    \item Non hai problemi a imparare Dart.
\end{itemize}

\textbf{Usa React Native se:}
\begin{itemize}
    \item Preferisci lavorare con JavaScript e React.
    \item Hai bisogno di una community ampia e molte librerie di terze parti.
    \item Vuoi un’integrazione più semplice con codice nativo.
\end{itemize}

\section{Conclusione}
Entrambi i framework sono ottimi strumenti per lo sviluppo mobile. La scelta dipende dalle esigenze del progetto e dalle competenze del team. 

\textbf{Se cerchi performance elevate e UI coerente → scegli Flutter.}

\textbf{Se vuoi flessibilità e facilità di sviluppo con JavaScript → scegli React Native.}

\end{document}