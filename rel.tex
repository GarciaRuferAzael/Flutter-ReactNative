\documentclass[a4paper,10pt]{article}
\usepackage[utf8]{inputenc}
\usepackage{graphicx}
\usepackage{hyperref}
\usepackage{booktabs} % per tabelle più professionali
\usepackage{titlesec} % per formattare titoli
\usepackage{tocbibind} % per includere l'indice
\usepackage[a4paper,margin=1in]{geometry}
\titleformat{\section}{\Large\bfseries}{}{0em}{}

\title{\textbf{Flutter vs React Native}}
\author{Azael Garcia Rufer}
\date{\today}

\begin{document}
\maketitle

\tableofcontents
\newpage

\section{Introduzione}
Negli ultimi anni, lo sviluppo di applicazioni mobili si è evoluto grazie a framework cross-platform come \textbf{Flutter} (Google) e \textbf{React Native} (Meta/Facebook). Questi strumenti permettono di creare app per Android e iOS con un’unica codebase, semplificando il processo di sviluppo.

Questa relazione si propone di:
\begin{itemize}
    \item Presentare Flutter e React Native, analizzandone vantaggi e svantaggi.
    \item Confrontarli in termini di performance, architettura e usabilità.
    \item Fornire indicazioni pratiche su quale scegliere in base alle esigenze del progetto.
\end{itemize}

\section{Flutter}
\subsection{Cos’è Flutter?}
Flutter è un framework open-source sviluppato da \textbf{Google} nel 2017. Utilizza il linguaggio \textbf{Dart} e un proprio motore di rendering chiamato \textbf{Skia}.

\subsection{Caratteristiche principali}
\begin{itemize}
    \item \textbf{Rendering Proprio}: Non usa i componenti nativi, ma disegna la UI con il proprio motore grafico.
    \item \textbf{Hot Reload}: Permette aggiornamenti istantanei senza riavviare l’app.
    \item \textbf{UI Personalizzabile}: Supporta Material Design (Android) e Cupertino (iOS).
    \item \textbf{Alta Performance}: Compilazione nativa (AOT - Ahead Of Time).
\end{itemize}

\subsection{Vantaggi e Svantaggi}
\textbf{Vantaggi:}
\begin{itemize}
    \item Prestazioni elevate e animazioni fluide.
    \item Interfaccia coerente tra iOS e Android.
    \item Supporto per Web, Desktop e Mobile.
\end{itemize}
\textbf{Svantaggi:}
\begin{itemize}
    \item App più pesanti rispetto a React Native.
    \item Dart meno diffuso rispetto a JavaScript.
\end{itemize}

\section{React Native}
\subsection{Cos’è React Native?}
React Native è un framework sviluppato da \textbf{Meta (Facebook)} nel 2015. Utilizza \textbf{JavaScript} e il framework React per creare app mobili sfruttando componenti nativi.

\subsection{Caratteristiche principali}
\begin{itemize}
    \item \textbf{Basato su JavaScript}: Facile da apprendere per chi conosce il web.
    \item \textbf{Componenti Nativi}: Usa gli elementi della UI propri di iOS e Android.
    \item \textbf{Hot Reload}: Permette di testare rapidamente le modifiche.
\end{itemize}

\subsection{Vantaggi e Svantaggi}
\textbf{Vantaggi:}
\begin{itemize}
    \item Ampia community e molte librerie disponibili.
    \item Facile integrazione con codice nativo.
    \item JavaScript è molto popolare tra gli sviluppatori.
\end{itemize}
\textbf{Svantaggi:}
\begin{itemize}
    \item Performance inferiore rispetto a Flutter (a causa del bridge JavaScript-to-Native).
    \item Problemi di compatibilità tra versioni di iOS e Android.
\end{itemize}

\section{Confronto Diretto}
\begin{table}[h]
    \centering
    \begin{tabular}{lcc}
        \toprule
        \textbf{Caratteristica} & \textbf{Flutter} & \textbf{React Native} \\
        \midrule
        \textbf{Linguaggio} & Dart & JavaScript (React) \\
        \textbf{Rendering UI} & Skia (personalizzato) & Componenti nativi \\
        \textbf{Performance} & Alta (compilazione AOT) & Buona, ma dipende dal bridge \\
        \textbf{Facilità di apprendimento} & Media & Facile (JavaScript è popolare) \\
        \textbf{Community} & In crescita & Più ampia \\
        \textbf{Ecosistema} & In espansione & Maturo e consolidato \\
        \bottomrule
    \end{tabular}
    \caption{Confronto tra Flutter e React Native}
\end{table}

\section{Flutter vs React Native – Qual è il migliore?}
La scelta tra Flutter e React Native dipende da vari fattori come i requisiti del progetto, le competenze del team di sviluppo e le necessità di performance.

Flutter, che utilizza Dart, offre alte prestazioni ma ha una base di sviluppatori più piccola e un supporto IDE limitato. React Native, che sfrutta JavaScript, ha un supporto comunitario più ampio e uno sviluppo più veloce grazie al linguaggio familiare.

Alla fine, la decisione dovrebbe essere basata sulle necessità specifiche del progetto e sulle preferenze del team di sviluppo.

\subsection{Perché scegliere Flutter}
\begin{itemize}
    \item Usa il linguaggio Dart, che compila più velocemente di JavaScript, garantendo alte prestazioni.
    \item Dart ha una diffusione inferiore rispetto a JavaScript, rendendolo meno familiare agli sviluppatori.
    \item Supporto IDE limitato per Dart a causa della sua minore popolarità.
    \item Consente uno sviluppo più rapido ma porta a file di dimensioni maggiori.
    \item Offre funzionalità di test integrate per facilitare i test.
\end{itemize}

\subsection{Quando scegliere Flutter}
Flutter può essere la scelta ideale quando:
\begin{itemize}
    \item La tua app ha bisogno di design UI altamente personalizzati e perfetti a livello di pixel. Flutter ha un'architettura basata su widget e funziona senza fare affidamento su componenti specifici della piattaforma.
    \item La tua app ha una grafica pesante e animazioni complesse e richiede alte prestazioni. Flutter può supportare questo tramite il proprio motore di rendering Skia.
    \item Vuoi che la tua app abbia un aspetto identico su dispositivi Android e iOS, quindi Flutter è la scelta giusta.
\end{itemize}

\subsection{Perché scegliere React Native}
\begin{itemize}
    \item Sfrutta JavaScript, che ha un supporto comunitario più ampio ed è più facile da imparare per gli sviluppatori.
    \item Prestazioni più lente a causa del bridge JavaScript-to-Native.
    \item Risparmia tempo per gli sviluppatori grazie all'uso di un linguaggio familiare.
    \item Richiede framework di test di terze parti come Detox per i test.
    \item Più ampiamente adottato e supportato da una comunità più grande.
\end{itemize}

\subsection{Quando usare React Native}
Conviene utilizzare React Native quando:
\begin{itemize}
    \item Vuoi riutilizzare i componenti della tua app desktop o sito web per una mobile app.
    \item Se il tuo team è più esperto in JavaScript o React, React Native dovrebbe essere la scelta ovvia, poiché utilizza lo stesso linguaggio e gli stessi concetti.
    \item Se desideri ridurre il consumo di memoria, in particolare sui dispositivi Android, React Native è una buona scelta, poiché il motore Hermes aiuta a ottimizzare l'uso della memoria.
    \item Quando la tua app dovrà fare un uso frequente di moduli nativi come fotocamera, GPS, ecc., React Native può essere scelto grazie alla sua robusta integrazione con le API native.
\end{itemize}


\section{Conclusione}
Entrambi i framework sono ottimi strumenti per lo sviluppo mobile. La scelta dipende dalle esigenze del progetto e dalle competenze del team. 

\textbf{Se cerchi performance elevate e UI coerente → scegli Flutter.}

\textbf{Se vuoi flessibilità e facilità di sviluppo con JavaScript → scegli React Native.}

\end{document}